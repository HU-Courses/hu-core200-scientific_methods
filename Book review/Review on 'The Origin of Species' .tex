\documentclass{article}
%\usepackage{apacite}
\usepackage{hyperref}
\usepackage{blindtext}
\usepackage{enumitem}
\usepackage{xcolor}

% Header and footer.

\title{\textbf{Habib University}\\ \textbf{CORE 200 - Scientific Methods}\\ \textbf{Book review}}
%\author{\textbf{Emad Bin Abid}}


\begin{document}
\maketitle

%\bibliographystyle{apacite}

\section*{About the Book:}
	Following are the necessary particulars of the book under review. \\
	\begin{description}[font=$\bullet$~\normalfont\scshape\color{red!50!black}]
		\item [Book] The Origin of Species
		\item [Originally published] November 24, 1859
		\item [Author] Charles Darwin
		\item [Page count] 502
		\item [Country] United Kingdom of Great Britain and Ireland
		\item [Genres] Treatise, non-fiction
		\item [Subjects] Natural selection, evolutionary biology
	\end{description}
 

\section*{About the Author:}
	Charles Robert Darwin (February 12, 1809 to April 19, 1882) was a naturalist and biologist known for his theory of evolution and the process of natural selection. Born in Shrewsbury, England, in 1831 he embarked on a five-year survey voyage around the world on the HMS \textit{Beagle}; his studies of specimens led him to formulate his theories. Charles Darwin is best known for his work as a naturalist, developing a theory of evolution to explain biological change. In 1859, he published \textit{The Origin of Species}. 
	
\section*{About the Reviewers:}
	This review is generated as a result of joint efforts of four students of Habib University; namely Emad Bin Abid, Saman Gaziani, Abeera Tariq and Muhammad Hanzila Jugnu.
\pagebreak

\section*{Abstract:}
	This review critically analyzes Charles Darwin's book; The Origin of Species. A thorough visual and textual exploration of this book offers the reader a unique interpretation of Darwin's evolutionary work on \textbf{natural selection}. This book primarily revolves around Darwin's research, correspondence and experimentation regarding the claim that the evolution of species over generations results due to a process of natural selection. The review will evaluate the author's scientific approach to, as well as the effectiveness of his collective outlook combined with varying types of theories on the issue.

\section*{Introduction:}
	This review is an analysis and critique on the book. The evolution of species is the one of the greatest subjects of research of all times. Even today, huge research projects are undergoing which aim to produce a sound answer of how the actual revolution happened. What role did mass extinctions play in the birth of new species and in the growth of remaining. Charles Darwin is one of those persons who tried to find the answers of evolution of species. His work is commendable and one of the master pieces in the history of evolution literature.

\section*{Analysis and Critique:}
	Charles Darwin builds his way towards explaining the phenomenon of evolution and the factors that influenced this process. It is intriguing to know more about this spectacle and understanding how certain conditions play a vital role in a species' existence and development. It all begins by pure observation and sets to advance in forming a conclusion. Darwin began by criticizing other claims such as contribution of the environment as suggested by other scientists to explain the variations in species however, accepted the involvement for distinguishing species. Darwin signifies \textbf{\textit{heredity}} as the crucial aspect as variations channel through to the descendants and the genetic behavioral and physical change is observed. We can easily note the phenomenon ourselves by observing the changes in our parents and ancestors. However, not every inheritance can be explained and ambiguity about why a certain pattern travels pertains. Over time, humans have participated in \textbf{\textit{conscious selection}} by breeders who choose to breed the most valuable species over others resulting in certain species to out-weigh the other or the same happens naturally by \textbf{\textit{unconscious selection}} where the breeder is not involved, yet the species evolve in a certain way.  The process adds traits to species that develop adaptations allowing them to cope with disastrous events. Darwin performs experiments and explain using scientific methodology to unravel unexplained phenomenon however, realizing that science cannot explain everything and sometimes you just have no answer and hit a dead end in the process of scientific inquiry. \\ \\
	Putting forward the concept of selection, he presents the ideology of \textbf{\textit{natural selection}} which identifies that organisms that better blend in their environment have higher survival and reproduction rate. To build a foundation for this concept, Darwin is appalled to learn naturalists of his time are unaware of the number of species in existence hence, leading to uncertainties in the classification methods. Some species tend to be more dominant and be part of a larger genera and such species are subjected to a higher probability of variation. It is of particular interest to discuss that the widely accepted idea of Darwin's time which suggested that species were created by God independently was rejected by these results hence, implicitly challenging Christianity. \\ \\
	Darwin makes intelligent use of Scientific methods of inquiry and witness the variations in species (of birds for example) therefore, analyzing the causes of varying species. The results to his field visit to the Galapagos Islands, contributed to the explanation of individual differences as geographical isolation is a prevailing factor and helps lead to the \textbf{\textit{struggle for existence}}. Struggle for existence and natural selection combine together to elucidate the creation of distinct species. Darwin uses examples to add weight to his theory. Organisms are able to adapt as the beak of the woodpecker assists in eating, the body of the parasite allows it to cling on other bodies for food and survival. It is therefore the beauty of nature that every specie has its own way to fight its war and survive in the competitive world. Where nature can help ease survival by providing necessities, natural hazards can result in a complete wipe out. Climate too plays a vital role in this development. Inspired by Malthus, Darwin discusses the exponential increase in population implying a higher rate of reproduction over years however a balance is ensured. The more important struggle is the struggle between species of the same kind. The specie that undergoes advantageous variation has a greater chance at life as it grows a shield against threatening circumstances. \\ \\
	Darwin's theories develop on observations which even we can easily perform. It is evident that species in dominant numbers tend to survive longer as there is minimal chance of their extinction due to predator attacks as they continue to reproduce. On the contrary smaller populations are in danger and hence, expected not to survive for a long period. Alongside crediting the work of fellow scientists such as Malthus, Darwin has maintained a graceful attitude towards explaining the content of his theories. Although his theories widely contradict with nature and its doings, Darwin highlights the beautiful qualities of nature that allow species to adapt whilst also portraying the negatives. Nature brings together species as friends and nemesis in the road to survival. Dependencies and threats both allow one to live in the environment. Darwin depicts the importance of competitive interactions that allow flourishing of species by giving the example of a tree planted in bare only to allow blooming of more species. \\ \\
	The idea thrown onto the table is later discussed by Darwin as the prime aspect of his theory to reason evolution, \textbf{\textit{natural selection}}.  Darwin makes his theory very vivid and clear however, leaving some questions unanswered. The theory lies on the assumption that the new variations are beneficial and hence, lay effect on the traits of the species as reproduction happens. Even though there is existence of ambiguity in the theory but this does not overshadow the brilliance of evolutionary history of species. Darwin suggested that the process of natural selection is gradual and it is significant to realize that over lifetime of a single human the change may be unnoticeable. Continuing to use work of other scientists to support his argument, Darwin relates the readers to the \textit{Lyell's Principles of Geology (1833)} which lays emphasis on the gradual change in geology of the world. \\ \\
	The use of metaphors is remarkable and endures a beautiful effect. Introduction of a new species in the geological environment is referred to as \textbf{\textit{immigrants}}. Further dwelling on natural selection and revisiting the involvement of breeders to allow certain species to have advantage over another, nature's involvement can increase an advantage and allow them to perpetuate through inheritance and ensure better survival. \\ \\
	Furthermore, Darwin talks about sexual selection which talks about variations in male species that allow females to be more attracted to them and hence resulting in higher rates of reproduction. This in turn is believed to be channeled to the offspring. This idea by Darwin sidelines the possibility of females consisting such characteristics that make them attracted to male species. Moreover, this concept rejects the chances of inheritance and passing down of genes from female species to the next generation therefore, Darwin's study reflects gender biasness. This may be due to the time period this study was being made, but scientifically a whole aspect of research has been set aside. \\ \\
	Variations result in the child species to diverge in terms of characteristics from the parent specie. These variations do not happen by mere luck but are linked to the geographical environments and other factors as presented by Darwin earlier. One possible reason that is brought into light is the use of organs that modify traits but this must be inherited for natural selection to be highlighted. Darwin uses the laws of variability to falsify the claim that species were created independently. Science is deeply linked with falsifiability and Charles Darwin has made brilliant use of this method. He says if species were independently created, all organs would be equally susceptible to variation. While making efficient use of inductive reasoning, Darwin is able to explain the laws of variation but it appears that he himself is slightly confused with his own theory hence as reader I am not completely convinced about the mechanism of variations. \\ \\
	Building upon the non-convincing attitude in Darwin's theory, there remain some unanswered questions in the theory. Darwin discusses on two of such questions but is not completely sure about the counter arguments provided by himself. He attempts to answer those questions on the basis of assumptions. Firstly, if the new species are emerged as a result of gradual transitions of nature, then why do we regard them as separate species instead of naming them as the intermediate descendants of older species? Secondly, can the transitions of nature over the period over many years produce complex organs within the species? \\ \\
	Darwin argues that the formation of a whole new separate specie is on the fact that the intermediate species between the early ancestors and the new species have become extinct due to the process of natural selection. This may happen due to various reasons. The major contributors for this natural phenomenon can be named as climate change, geographical change, and the overall environmental change. These intermediate species were not able to reproduce appropriately because of the reasons mentioned above. Therefore, they became extinct and are not remembered now. Only their descendants who able to resist to the natural climaxes and reproduce sufficiently exist as separate species. \\ \\
	Darwin is unable to answer satisfactorily to his second question. He is unable to explain the major development of complex organs in the species. He could neither answer how the eye had developed in species nor was he able to address the question of wings on a bat. But Darwin still provided some clue that these changes had developed over the period of many years as a result of change in orientation of nerves in a body or maybe sometimes the development of new nerves in it. \\ \\
	One thing that is appreciable in Darwin's work, which is also reflected in this book, is the clear distinction between \textbf{\textit{habit}} and \textbf{\textit{instinct}}. He discusses the role of natural selection in the development of instinct. He discusses that habit is something which a certain creature can learn. On the other hand, instinct is inherited. It might be possible that a habit of an ancestor gets inherited as an innate instinct in a descendant specie. \\ \\
	Darwin also gives some examples of instinctive abilities in species. For instance, the bees have an innate instinct which leads them to develop the honey comb in such a geometry that the comb holds maximum amount of honey. The hens have an innate ability that they lay eggs only in the hen's nest which allows them to give birth to as many baby hens as they wish without worrying about to take care of all of them.  \\ \\
	Going ahead in the later sections of the book, Darwin mentions his theoretical results about the sterility of certain species. He names these kinds of species as hybrid species. There are times when the parent species are not able to conceive properly due to their natural properties. When this happens, they result in giving birth to hybrid species. These hybrid species themselves are not able to reproduce as their reproductive systems are not suitable enough to conceive properly either. This leads to the idea of development of sterile species. \\ \\
	With this being said, Darwin also neglects the idea of scientists who say that if certain species are enormous in number, then it must be the fact that they are the off-springs of those species who have conceived with the same species. Darwin also introduces a new term called \textbf{\textit{systematic affinity}}. Systematic affinity is the ability of a certain specie to produce a fertile off-spring due to physical and structural similarities. \\ \\
	After studying it deeply, Darwin's work is majorly based on assumptions. We cannot find extensive practical proofs of whether or not the true facts about species go hand-in-hand with those provided by Darwin. Similar is the case with the critics of the then period when this book was written. But Darwin provides counter answers to the critics of his theory. The critics believe if Darwin's theory is true then it must go hand-in-hand with the geological record. The land must provide clues and answers to the relationship between existing species and the then existing parent species. But, in practical, it is not the case. There have been very few instances when the scientists have been able to find connection between the two. Darwin counters the scientists by mentioning that the non-availability of geological evidence is in-sufficient to discard his theory of natural selection. He argues that the fossil is not the same as it was years ago. There have been environmental and geological changes which might have resulted in erasing the evidence. \\ \\
	Darwin believed that there are some \textbf{\textit{geographical barriers}} which affect the distribution of species and classification of species in a particular location. According to Darwin, climate conditions and migration opportunities are two of many reasons which shape the distributions of species across the continents. In parts of land where migration is possible between them, it can be noticed that the species in these parts are similar to each other and there are higher chances of crossbreeding and origination of new species. This division between parts does not just limit to land but also affects the separated bodies of water and the barrier still influence the distribution. Within continents or even sea, species are connected to each other directly or indirectly. Darwin believed that species originated from a single place and then spread to other parts of land due to migration or other methods. These other methods, as described by Darwin, includes the transfer through birds and icebergs in case of plant species. \\ \\
	Darwin points out two geological changes through which he is able to deduce some pattern in spread of species. Ice Age being one of them, where the climate was uniform throughout and only specific species could survive in these conditions. The one who could survive in extreme cold weather migrated to the top of the mountains. And these migrations, as stated by Darwin, are the reason for the same species on the mountain tops till date. Another geological change in the history of world, Bering strait, a connection between areas stretching from Western Europe through Siberia to Eastern America. This part is known to be a land bridge in earlier time but due to rising sea level, the water covered this part. According to Darwin, before the area was covered with water, migration might have taken place by many species and this is the reason why analogous species appear between some parts of the world. Therefore, these affects the diverging natural specie selection in isolated parts. \\ \\
	Later, Darwin talks about how changes in land level has caused separation in fish species as well. Same fish species can exist in water parts of same continent but not in different continents. Darwin also points out that freshwater species can also migrate through ducks that carry shells from one water body to another. He noticed that the species found on islands are extremely different from the one found on continents. Island holds \textbf{\textit{endemic species}}, species that are not present anywhere else. Also, species that have capacity to migrate are less likely to change for adaptation. Darwin notes that geological barriers did not existed earlier and so there might be a way to migrate between two parts that are separated now. He further deduced that the species on islands are similar to those found in nearby continent, irrespective of the climate conditions. Similarly, if two parts of the world, with same climatic conditions, are disconnected, there is no way that both can have identical species due to no migration opportunities. \\ \\
	Concluding with his deduction, Darwin recognizes geological barriers and its limitations. He explains the reason of diverging species and the possibilities of no geological barrier in the past. He emphasized on the importance of migration that have resulted in the spread of species. He also discussed how climatic changes were important when the climate was same throughout and only the species that could adapt would survive. However, in the end, migration and survival are random, as we cannot keep track of the migration tracks and the barriers that were not present before. \\ \\
	Moving ahead, in the later sections of the book, Darwin focuses on the basis of categorization and division of species. He argues that the naturalists designed this classification system and he uses it to explain his theory of descent. Naturalist agree that external structures are not enough to classify species to genera, they rely on internal structures to observe the development patterns. \\ \\
	\textbf{\textit{Morphology}}, a study of relationship between physical organs of different species. They focus on the bones that are structured externally different but similar placement internally, to adapt physically to their habit. These show that how similar structures exist from different species. Darwin further discusses how development of embryos result in development of a new specie, different form its parent species. This development is to make the new specie adapt to its environment just like humans evolve. \textbf{\textit{Embryology}} is a dedicated study of embryos which researches use to reflect on how the species are diverging from its parent as time passes and classifying them to their families. \textbf{\textit{Rudimentary organs}} – organs that are left over by parent species and are not used for the same purpose now. These organs reveal relationship to their descent, as even if the organs are not used for the same purpose now, but still it exists at the same part of the body as before and can related to it. It violates the theory that each specie is created independently and perfectly, but the current species contain useless organs as well. 
Concluding this section of the book, it is quite true to say that Darwin did a brilliant job in linking his descent theory with the classification system developed by the naturalists. He said that the species can be classified into same category because they belong to the same descent. His argument against the theory of perfect development stands valid here. It can also be noted that Darwin compares human to moles and bats, supporting the idea that human have evolved from these species. \\ \\
	As Darwin concludes his master piece, he summarizes his work by answering the questions for infertility of hybrids, geological distribution and the fault with geological record. He admits that there are gaps in these theories that would be filled with time. He emphasized that differences exists in nature and species compete with each other for survival in this world which causes divergent in cause of many species. Darwin further argues that the naturalist are not ready to accept his theory of descent due to many reasons; naturalists were unable to grasp how the species diverge and how long does it take for a new specie to emerge was not yet answered. He said that with this book being shared, it would be a turning point for naturalist and scientist to start thinking about these theories. According to him, now naturalists would have a better understanding of the theory if descent and this would be a revolutionary time, where their focus would turn towards understanding of chains of descent and how humans are involved in this process. \\ \\
	On further analysis we find that Darwin agrees to the possibility of a \textbf{\textit{Creator}} who is responsible for the origin of species. His hypothesis about evolution of human beings gave birth to many questions and have been a concern for scientists since then.


\section*{Personal Statement:}
	After having a deep analysis of the book, we think that the book is worth reading. Even though we do not agree with some arguments made by Darwin but still the book provides an insight of different possibilities of what could have happened in the evolutionary stages. The book continuously probes the reader to dig in the argument made by Darwin and extract the best possible meaning out of the theory. Therefore, we recommend the audience to read this book as the readers may find some interesting and mind-opening scientific facts and arguments backed by Charles Robert Darwin. 
\pagebreak

\section*{References:}
	Towards the end, we would like to thank all our references which guided us throughout the process of review. The online article \textbf{\textit{How to Write a Scientific Book Review}} by Erin Schreiner (accessed on March 22, 2018) was the sole reference of formulation of entire document. The article can be accessed \href{http://penandthepad.com/write-scientific-book-review-7762473.html}{here}. \\ \\
	We would also like to thank an online biography on Charles Darwin (accessed on March 22, 2018) which provided us the sufficient information to be included as a part of this review. The biography can be accessed \href{https://www.biography.com/people/charles-darwin-9266433}{here}. 

%\bibliography{References}
\end{document}