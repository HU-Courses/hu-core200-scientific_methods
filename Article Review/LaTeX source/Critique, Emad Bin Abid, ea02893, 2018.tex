\documentclass{article}
\usepackage{apacite}

% Header and footer.

\title{\textbf{Habib University}\\ \textbf{CORE 200 - Scientific Methods}\\ \textbf{Research critique}}
\author{\textbf{Emad Bin Abid}}


\begin{document}
\maketitle

\bibliographystyle{apacite}

Jan Zalasiewicz, “A History in Layers”, Scientific American, Special Edition, Winter 2016, November 22nd, 2016.\\

\begin{center}
	Humans are profoundly altering the earth. Is our impact enough to matter
	across geologic time? Some say it is. Welcome to the Anthropocene.\cite{quote}
\end{center}

\section*{Origin of Article:}
The article under review has been extracted from the magazine \textit{Scientific American } from its winter edition of 2016. Magazine under discussion is not a peer-reviewed product but its recent coalition with the corporate family of Nature Publishing Group (a peer-reviewed scientific journal) adds more to its credibility and integrity. Moreover the magazine is under the administration of renowned and credible scientists of current era. The article is recently published and discusses on the main problem currently the globe is facing. It discusses some significant points which can further be utilized to lay the foundation for many research driven papers and articles in the domain of discussed topic. 

\section*{Theme of Research Study:}
The research study drives the attention of readers to the possible probability of sixth stage of earth’s evolution as proposed by one of the world’s most respected and popular scientists, Paul Crutzen. The author builds upon the arguments of Crutzen and advances his opinion that an era of Holocene has passed and now the world is advancing towards a completely new frame where humans have influenced the atmosphere and now the Earth is experiencing on of the drastic changes of all time. This era we now call as “Anthropocene”. The author is evidently critical in analyzing the factors which have seriously caused a damage to the environment on Earth. Hence claiming the evolution of sixth stage of the globe. 
	
\section*{Critical Analysis of Research Article:}
The geologic change since the industrial revolution (eighteenth century) is the main observatory channel and provides the lens to see through the reasons which have practically altered the nature of how the Earth behaved over the period of time. The melting of polar ice-caps resulting in an increase in sea-level globally provides the major evidence that something unpleasant has been going on which has drastically warmed the planet’s environment. The author firmly believes in the causes and effects of such situation but points out a significantly interesting research question. \textit{Will the changes made by us Humans last so long that the geologists studying the Earth’s environment after a million more years find the evidences of the sources which caused the environment to alter?} The author uses a word ‘epoch’ whenever he discusses the term anthropocene. Frequently, the word ‘Anthropocene’ had been widely used in various scientific and socio-scientific journals and articles which gave weight to the discussions on \textit{anthropocene; as a new evolutionary period}. In order to prove the authenticity of information flow regarding anthropocene as an advancement towards a new era of evolution, scientists must prove the long-lasting effects of the claims they are making. They must ensure that the factors which are bringing the changes the Earth’s climate and atmosphere must last for a great many number of years so that they may be explored and discovered even if the mankind gets extinct, some other beings take over the charge to rule over the Earth or the geologists living in that time can discover them. The author continuously uses the term ‘strata’ in his discussion. This is so much important to address as the humans living in the far future will definitely have the Earth’s layer of strata even if all the other resources of information or discovery get eliminated. The climatic changes, the resources of energy used, and much more are all somehow preserved in the rocks which can later be used as an information source for discoveries. The reason this research critique emphasizes upon the strata is more because today if we have knowledge of evolution of life, growth of trees, oceans, safe air to breathe, and more are all the secrets kept by the rocks in themselves. Elsewise no one would have even imagined how the surface of Earth, which kept burning millions of years after its creation, would contain three-fourth of its area filled with water currently. \\
The following main resources author has mentioned which must leave the footprints behind for the gain of past knowledge in the upcoming far future.
\subsection{\underline{Rocks and Ores:}} Earlier in time, minerals were developed due to natural processes and natural chemical reactions. Today, humans have used these minerals to further make different compounds which are of daily basis use for us. For instance, aluminum products are widely used today which is obtained by the extraction of aluminum metal from its ore. The question of research still holds whether these products will last for years and will they be sustainable enough to remain as an identifier of early age environment with respect to the far future. Human made synthetic rocks (bricks) are the main source of building mega-structures and the same rule holds for these too. 
\subsection{\underline{Chemical Fingerprints:}} In order to develop in technology sector, different chemical reactions are being conducted and some of those reactions leave harmful influences to the environment we are living in. Common examples can be found in burning of daily basis fossil fuels which leave enormous amount of carbon dioxide and other harmful side products into the atmosphere. Moreover, radioactive experiments conducted in the field nuclear technology have dangerous effects of its own kind. Since World War II, several nuclear weapons have been tested and the world has now been equipped with mega nuclear weapons. 

\subsection{\underline{Fossil Transitions:}} The humans have also left their traces on the Earth’s biological landscape. Today, humans are the most dominant creatures currently living on the planet. 

The debate still remains the same. The research question still upholds. \textit{Will these products last till the far future?}


\section*{Concluding Remarks:}
\textit{A History in Layers} is an informative research article published in Scientific American magazine. Since it is not a peer-reviewed magazine itself, we cannot totally accept the firm decisions made in the article but we have leverage to use this article as a sound scientific research work because the magazine is in coalition with one of the scientific journals (which is peer-reviewed). The article mainly discusses on the shift of era from Holocene to Anthropocene. It is true that humans today are benefitting themselves from the resources exposed to them at the fullest. We are not caring about the harms and damages our atmosphere is facing due to neglected safety measures. Modern day example of Anthropocene is the extinction of Ozone from atmosphere. Period of Anthropocene roughly began around 1950s when nuclear weapons were being developed and rapid growth was observed in the field of Chemistry. The research needs to be widened in exploring whether the causes of these rapid environmental changes last long enough up to millions and billions of years or not. On the whole, the research issue is worth for more detailed and knowledgeable discussion. 

\bibliography{References}
\end{document}